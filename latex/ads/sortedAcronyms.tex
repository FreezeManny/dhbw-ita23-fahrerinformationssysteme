%		\ac{Abk.}   --> fügt die Abkürzung ein, beim ersten Aufruf wird zusätzlich automatisch die ausgeschriebene Version davor eingefügt bzw. in einer Fußnote (hierfür muss in header.tex \usepackage[printonlyused,footnote]{acronym} stehen) dargestellt
%		\acf{Abk.}   --> fügt die Abkürzung UND die Erklärung ein
%		\acl{Abk.}   --> fügt nur die Erklärung ein
%		\acp{Abk.}  --> gibt Plural aus (angefügtes 's'); das zusätzliche 'p' funktioniert auch bei obigen Befehlen
%		\acs{Abk.}   -->  fügt die Abkürzung ein
%	siehe auch: http://golatex.de/wiki/%5Cacronym
%!TEX root = ../dokumentation.tex
%nur verwendete Akronyme werden letztlich im Abkürzungsverzeichnis des Dokuments angezeigt
%Verwendung: 
\acro{z.B.}{zum Beispiel}
\acro{CAN}{Controller Area Network}
\acro{LIN}{Local Interconnect Network}
\acro{ESP}{Elektronisches Stabilitätsprogramm}
\acro{ABS}{Antiblockiersystem}
\acro{Kfz}{Kraftfahrzeug}
\acro{EMV}{Elektromagnetische Verträglichkeit}
\acro{LSB}{Least Significant Bit}
\acro{LDF}{LIN Description File}
\acro{ISO}{International Organization for Standardization}