%!TEX root = ../dokumentation.tex

%
% vorher in Konsole folgendes aufrufen:
%	makeglossaries makeglossaries dokumentation.acn && makeglossaries dokumentation.glo
%
% oder makeglossaries dokumentation
%
% Um das auszuführen, braucht man ActivePerl oder ein ähnliches Programm. ActivePerl kann man bei MyIT Services beantragen

%
% Glossareintraege --> referenz, name, beschreibung
% Aufruf mit \gls{...}
%
% Glossar wird nur angezeigt, wenn Glossareintrag in Text referenziert wird
%
%\newglossaryentry{touch}{name={touch},plural={touch},description={Der \textit{touch} Befehl erstellt neue Dateien. Verscheidene Parameter kann man über \textit{touch -{}-help} oder \textit{\gls{man} touch} herausfinden}}
