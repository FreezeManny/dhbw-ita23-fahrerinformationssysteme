%!TEX root = ../dokumentation.tex

\chapter{Fazit}

Zusammenfassend kann man sagen, dass beide Bussysteme auf dem Markt eine Berechtigung haben.
CAN sticht vor allem wegen der schnellen Datenraten, Zuverlässigkeit, einfacher Erweiterbarkeit und der recht hohen Fehlersicherheit auf.
Außerdem werden wichtige Botschaften priorisiert, was eine reibungslose Kommunikation bei Echtzeitanwendungen gewährleistet. Dies kann allerdings dazu führen, dass es bei einem falsch konzipierten System zum Untergang von Nachrichten geringer Priorität kommen kann.
Im Gegensatz zu CAN, wird LIN hauptsächlich für unkritische Anwendungsbereiche verwendet, da LIN eine deutliche geringere Datenrate, Fehlerprävention und Fehlerüberprüfung hat. Dadurch kann ein LIN-Bus allerdings sehr kostengünstig und mit wenig Aufwand in einem Fahrzeug verwendet werden. 

Da man in einem Automobil unterschiedliche Anwendungsbereiche abdecken muss, werden unterschiedliche Bussysteme, wie LIN, CAN, MOST und Flexray, verwendet und durch Gateways miteinander vernetzt. 
Dadurch vereint man die Vorteile aller Bussysteme und sorgt für eine möglichst kostengünstige und effiziente Vernetzung in den verschiedenen Anwendungsbereichen im Automobil. 
