%!TEX root = ../dokumentation.tex

\chapter{Grundlagen}
\section{Definition und Entwicklung des Automotive Ethernet}
\section{Treiber für die Einführung im Automobilbereich}
\section{Standardisierung}
\subsection{IEEE 802,3}
%\subsection{BroadR-Reach}
\subsection{1XXXBase -T1}


\chapter{Technologische Unterschiede zum Standard-Ethernet}
\section{Leitungstechnologie: Ungeschirmte Einzelpaarverkabelung}
\section{Physikalische Schicht: EMV-Anforderungen und Signalcodierung}
\section{Automotive-spezifische PHYs und ihre Besonderheiten}

\chapter{Topologie und Netzwerkkonfiguration}
\section{Sterntopologie vs. Bus-Struktur des CAN}
\section{Auswirkungen auf die Fahrzeugarchitektur und Verkabelung}
\section{Switch-basierte Netzwerke und ihre Vorteile}

\chapter{Integration von CAN-Daten über Ethernet}
\section{Gateway-Konzepte für die CAN-Ethernet-Kommunikation}
\section{Protokollkonvertierung und Timing-Herausforderungen}
\section{Quality-of-Service für zeitkritische Anwendungen}

%\chapter{Praxisbeispiele und Anwendungsbereiche}
%\section{ADAS und autonomes Fahren}
%\section{Infotainment und Kamerasysteme}
%\section{Aktuelle Implementierungen in der Automobilindustrie}
%
%\chapter{Zukunftsperspektiven}
%\section{Multi-Gig Automotive Ethernet (802.3ch)}
%\section{Time-Sensitive Networking (TSN)}
%\section{Zonenarchitekturen und zentrale Computing-Plattformen}

\chapter{Fazit}
\section{Zusammenfassung der Hauptunterschiede}
\section{Bewertung der Vor- und Nachteile}
